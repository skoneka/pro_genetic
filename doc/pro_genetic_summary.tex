\documentclass[bibtotocnumbered, headsepline,normalheadings,12pt,polish]{scrreprt}
\usepackage[T1]{fontenc}
\usepackage[utf8]{inputenc}
\usepackage{geometry}
\geometry{tmargin=25mm,bmargin=25mm,lmargin=30mm,rmargin=30mm}
\usepackage{babel}
\begin{document}
\chapter*{Porównanie prototypów}
\begin{tabular}{| p{3cm} || p{6cm} | p{6cm} |}
\hline
 & Prototyp 1 & Prototyp 2 \\
\hline
\hline
Zastosowany algorytm & Breadth First Search ( przeszukiwanie wszerz ) & Circle Packing \\
\hline
Zastosowane narzędzia & Java, Swing & Python, PyGame \\
\hline

Zasada działania funkcji oceny &
$ (wygoda-koszt)*bfs/dlugosc $
\begin{verbatim}
Miejsce   siedzące   stojące
Wygoda    7          3
Koszt     3          1
\end{verbatim}
&
$ komfort + koszt + pojemność $
\begin{itemize}
\item komfort ( liczba siedzeń )
\item koszt ( przesteń zajmowana przez siedzeń i wolna przestrzeń )
\item pojemność ( całkowita liczba pasażerów w autobusie )
\item wagi  ( Ffactor-comfort, Ffactor-cost, Ffactor-capacity)
\item ceny ( Fseat-cost, Fspace-cost )
\item wartości minimalne i maksymalne
\end{itemize}
\\
\hline
Kodowanie genotypu & \begin{itemize}
    \item szerokość w liczbie mieszczących się w kolumnie ludzi
    \item  długość w liczbie mieszczących się w rzędzie ludzi
    \item rozmieszczenie siedzeń i miejsc stojących
\end{itemize}
&
\begin{itemize}
    \item długość
    \item szerokość
    \item maksymalna liczba miejsc siedzących
    \item liczba rzędów
    \item odległość między rzędami
    \item parametr określający co ile siedzeń w rzędzie występuje separator
\end{itemize}
\\
\hline
\end{tabular}


\begin{tabular}{| p{3cm} || p{6cm} | p{6cm} |}
\hline
 & Prototyp 1 & Prototyp 2 \\
\hline
\hline
Selekcja osobników &
\begin{itemize}
    \item metoda rankingowa 
    \item elitizm
\end{itemize}
&
\begin{itemize}
    \item ruletka 
    \item elitizm
\end{itemize}
\\
\hline
Krzyżowanie osobników &
\begin{itemize}
\item Losowanie osobnika w którym zapisana jest pozycja siedzenia.
\item Podział autobusów na 3 kolumny i wymieszanie z drugim osobnikiem.
\end{itemize}
&
\begin{itemize}
\item Łączenie genotypów na ``suwak''
\end{itemize}
\\
\hline
Mutacja & Jedno z miejsc ulego transformacji (osobniki wg. prawdopodobieństwa) & na losowej liczbie pozycji zostaje przypisana losowa wartość ( osobniki wg. prawdopodobieństwa ) \\
\hline
Szybkość działania & Wydajny obliczeniowo & Potrzeba dość dużej mocy obliczeniowej podczas symulowania większej liczby pasażerów. \\
\hline
Wyniki & Zastosowanie BFS umożliwia programowi ustalenie ścieżek dostępu do wszystkich siedzeń nawet przy skomplikowanym układzie autobusu. & Dzięki zastosowaniu algorytmu Circle Packing możliwe staje się dość dokładne ustalenie liczby osób która jest w stanie zmieścić się w autobusie. \\
\hline
\end{tabular}
\end{document}
\end{document}

